\cleardoublepage

\chapter*{Agradecimientos}

Primero de todo me gustaría agradecer a mi tutor Julio toda la confianza y el apoyo depositado para poder hacer posible este proyecto. Sin ti nada de esto hubiera sido posible.

También agradecer a todos los profesores que he tenido a lo largo de la carrera lo importante que habéis sido para mí tanto en mi crecimiento académico, profesional como personal. Gracias a vosotros me habéis hecho ingeniera.

Otras personas que han sido, son y serán fundamentales en mi vida son mis padres; a los cuáles estaré eternamente agradecida por no soltarme nunca de la mano y darme todas las herramientas que han hecho posible ser quien soy hoy en día. Os quiero muchísimo.

Una persona que apareció en mi vida hace ya unos 7 años fue Fran, quien trajo luz en medio de tanta oscuridad y ha sido el mejor acompañante, amigo y amante de cada aventura que se presenta en la vida. Te amo con todo mi corazón y gracias de verdad por todo, ya lo sabes.

Los abuelos deberían ser eternos pero desgraciadamente en mi caso marcharon mucho antes de lo debido. Sé que les encantaría ver a su nieta graduada como ingeniera y me encantaría poder darles un abrazo muy fuerte y poder celebrarlo juntos pero bueno, se que allá donde estéis estaréis muy orgullosos de mí tanto como yo os echo de menos a vosotros. Gracias por los años que pude conoceros y disfrutaros, quedarán para siempre en mi retina. 

Los amigos son la familia que uno puede elegir y hoy en día puedo estar muy orgullosa y agradecida de la gente que tengo a mi lado. Gracias a Jimena, Elisa, Sofía, Maryu, Santi, Gonzalo y Álvaro por ser tan maravillosos y por estar ahí estos 4 años de alegrías, lloros, mucho trabajo y dolores de cabeza; sois increíbles. Gracias también a Ángela, Roberto y Lío por aparecer antes de esta aventura y por seguir estando ahí contra viento y marea. Os quiero mucho a todos. \\
\ % Algo de separación...

\

\

\

\

\begin{flushright}
		\vspace{4.0 cm}
		\emph{A mi misma;\\
      por nunca tirar de la toalla}\\
		\par
		\vspace{1.0 cm}
		Chozas de Canales, 26 de Diciembre de 2024\\ %\today CAMBIAR
		\emph{Julia López Augusto}
\end{flushright}

\thispagestyle{empty}

