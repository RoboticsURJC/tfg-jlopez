\cleardoublepage

\chapter*{Abstract\markboth{Abstract}{Abstract}}


%La robótica es el campo de la ingeniería que se enfoca en el diseño, la construcción y la programación de robots. De todos los tipos de robots que existen cabe destacar los robots de campo, que son capaces de operar en entornos no estructurados y difíciles, pero estos son de elevado coste. Es por ello que, bajo el prisma de la robótica \textit{low cost}, campo creado para hacer robots económicos, accesibles y fáciles de producir, se puede idear una solución asequible a este problema.

Robotics is the field of engineering focused on the design, construction and programming of robots. Among the various types of robots, field robots stand out for their ability to operate in unstructured and challenging environments. However, these robots are often expensive. For this reason, under the point of view of low cost robotics, a field created to develop economical, accessible and easy to produce robots, an affordable solution to this problem can be devised.

%Una de las aplicaciones de los robots de campo es el mantenimiento de carreteras. El presente trabajo pretende solucionar este problema diseñando un robot de bajo coste que permita la detección de baches, almacene su localización vía GPS, y calcule el volumen de estos. Además, se ha desarrollado toda una interfaz web, para que la persona de mantenimiento pueda gestionar el mantenimiento de forma cómoda e intuitiva.

One of the applications of field robotics is road maintenance. This work aims to address this issue by designing a low cost robot that allows the detection of potholes, stores their location via \acs{GPS}, and calculates their volume. Moreover, a complete web interface has been developed to enable the maintenance service to manage maintenance tasks in a convenient and intuitive way.

%El prototipo robótico se ha diseñado en 3D, para lo que se han empleado herramientas de \textit{software} libre. Su impresión se ha hecho en una impresora convencional, y sus componentes \textit{hardware} de terceros se pueden encontrar fácilmente en cualquier tienda de electrónica. Además, se ha desarrollado una versión de este robot en simulación, al que se le ha dado soporte en ROS 2. 

The robotic prototype has been designed in 3D using open-source software tools. It has been printed on a conventional printer, and its third-party hardware components are easily available at any electronics store. Additionally, a simulated version of this robot has been developed, with support in ROS 2.

%Algunas de las técnicas empleadas en el robot físico, para conseguir el objetivo principal, son: el aprendizaje supervisado, \ac{TPU}, filtros de imagen, el modelo pinhole, el algoritmo de la lazada, \ac{VFF}, servidores web, entre otros.

Some of the techniques used in the physical robot to achieve the main objective, are: supervised learning, \ac{TPU}, image filters, the pinhole model, the shoelace method, \ac{VFF}, web servers, among others.

%Numerosos son los experimentos realizados, tanto para evaluar cada componente de forma individual, como para afinar el comportamiento completo del robot. Uno de los comportamientos supone manejar el robot de forma teleoperada y, el otro, que el robot se mueva de forma autónoma, usando una carretera adaptada como su entorno de operatividad.

Numerous experiments have been carried out, both to evaluate each component individually and to fine-tune the robot's overall behavior. One of the behaviors involves operating the robot via teleoperation, while the other allows the robot to move autonomously, using a adapted road as its operational environment. 