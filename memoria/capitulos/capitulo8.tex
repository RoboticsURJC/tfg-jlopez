\chapter{Conclusiones}
\label{cap:capitulo8}

\begin{flushright}
\begin{minipage}[]{10cm}
\emph{La mente lo es todo. En lo que piensas, te conviertes}\\
\end{minipage}\\

Buda\\
\end{flushright}

\vspace{1cm}

En este último capítulo se van a detallar todos los objetivos y subobjetivos cumplidos; así como las competencias adquiridas a lo largo de todo el proyecto. Finalmente se concluirá la memoria exponiendo las posibles líneas futuras para el proyecto.


\section{Objetivos cumplidos}

En este apartado se va

%El objetivo principal era: 

%Crear un robot que, usando materiales de bajo coste, sea capaz de navegar por las carreteras, detectar los baches que vaya encontrando y sea capaz de estimar el área del bache para hacer una estimación media del volumen que ocupa dicho bache y poder ser tapado. De igual manera, todo quedará registrado en una interfaz web en la que cada bache quedará marcado sobre un mapa con su correspondiente descripción para que los operarios puedan operar cuando estimen oportuno.


No olvides dedicar un par de párrafos para hacer un balance global de qué has conseguido, y por qué es un avance respecto a lo que tenías inicialmente. Haz mención expresa de alguna limitación o peculiaridad de tu sistema y por qué es así. Y también, qué has aprendido desarrollando este trabajo.\\

% Los subobjetivos:

%1. Investigar los robots o soluciones actuales que cumplan con las caracterı́sticas y objetivos establecidos.
%2. Seleccionar los componentes hardware de bajo coste necesarios para construir el esqueleto del robot.
%3. Analizar las diferentes opciones de diseño que más encajen con la forma del robot.
%4. Diseñar las piezas en CAD usando herramientas de software libre.
%5. Usar material tı́pico de impresión 3D para imprimir las partes del robot, como puede ser ABS o PLA.
%6. Desarrollar un modelo del robot para que pueda ser usado en simulación usando herramientas robóticas.
%7. Desarrollar software necesario usando herramientas robóticas para poder controlar el robot fı́sico.
%8. Realizar algunos experimentos en entorno reales o adaptados.

%% Terminar de razonar estos subobjetivos 

\section{Competencias adquiridas}

Las competencias adquiridas a lo largo del proyecto

conocimientos profundos/skills de Freecad (ampliado experiencia)

He aprendido sobre estructuras y diseños

He podido trabajar con la Raspberry usando puerot CSI, Serial (configurado y probado)  que previamente no se había hecho

He conseguido implementar un modelo de aprendizaje supervisado sobre un dispositivo que tenía limitciones computacionales

He aprendido en muy profundidad el funcioanmiento del modelo pin hole 

He aprendido sobre cómo investigar y los pasos a seguir 

He aprendido a generar documentación de calidad para un trabajo al igual que conocer sus etapas y saber respetarlas 

He aprendido nuevas técnicas matemáticas 

He aprendido sobre cómo usar una interfaz web 

He aprendido a integrar ROS2 a un dispositivo los limitaciones

He aprendido a construir un robot definido por urdf/xacro

He aprendido a integrar  una arquitectura ros2 control en un robot

\section{Líneas futuras}


Por último, añade otro par de párrafos de líneas futuras; esto es, cómo se puede continuar tu trabajo para abarcar una solución más amplia, o qué otras ramas de la investigación podrían seguirse partiendo de este trabajo, o cómo se podría mejorar para conseguir una aplicación real de este desarrollo (si es que no se ha llegado a conseguir).


\begin{itemize}
	\item \textit{Integración en la vida real del motor de la cámara} Aumenta la complejidad...
	\item \textit{Aumentar el campo de visión de la cámara} Haciendo más largo el soporte
	\item \textit{Desarrollar software para el robot simulado} Para que las caídas no duelan
	\item \textit{Implantar PiBotJ como ayudante de mantenimiento} primero en pueblos y más adelante en carreteras
	\item \textit{Continuar dando soporte a PiBotJ} para mantenerlo actualizado en las últimas versiones
\end{itemize}\

