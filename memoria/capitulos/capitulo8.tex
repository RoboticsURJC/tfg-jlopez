\chapter{Conclusiones}
\label{cap:capitulo8}

\begin{flushright}
\begin{minipage}[]{10cm}
\emph{La mente lo es todo. En lo que piensas, te conviertes}\\
\end{minipage}\\

Buda\\
\end{flushright}

\vspace{1cm}

Escribe aquí un párrafo explicando brevemente lo que vas a contar en este capítulo, que básicamente será una recapitulación de los problemas que has abordado, las soluciones que has prouesto, así como los experimentos llevados a cabo para validarlos. Y con esto, cierras la memoria.

En este último capítulo .... objetivos cumplidos y conclusiones derivadas y líneas fututras 

incluir competencias adquiridas 

Se cerrará el TFG...

\section{Objetivos cumplidos}

Enumera también los requisitos implícitos en la consecución de esos objetivos, y cómo se han satisfecho.\\

\subsection{Objetivo principal}
El objetivo principal era: 

"

\subsection{Subobjetivos}

No olvides dedicar un par de párrafos para hacer un balance global de qué has conseguido, y por qué es un avance respecto a lo que tenías inicialmente. Haz mención expresa de alguna limitación o peculiaridad de tu sistema y por qué es así. Y también, qué has aprendido desarrollando este trabajo.\\

\section{Conclusiones finales y líneas futuras}


Por último, añade otro par de párrafos de líneas futuras; esto es, cómo se puede continuar tu trabajo para abarcar una solución más amplia, o qué otras ramas de la investigación podrían seguirse partiendo de este trabajo, o cómo se podría mejorar para conseguir una aplicación real de este desarrollo (si es que no se ha llegado a conseguir).


\begin{itemize}
	\item \textit{Integración en la vida real del motor de la cámara} Aumenta la complejidad...
	\item \textit{Aumentar el campo de visión de la cámara} Haciendo más largo el soporte
	\item \textit{Desarrollar software para el robot simulado} Para que las caídas no duelan
	\item \textit{Implantar pibotj como ayudante de mantenimiento} primero en pueblos y más adelanteen carreteras
\end{itemize}\

