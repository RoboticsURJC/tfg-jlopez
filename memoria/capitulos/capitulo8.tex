\chapter{Conclusiones}
\label{cap:capitulo8}

\begin{flushright}
\begin{minipage}[]{10cm}
\emph{La mente lo es todo. En lo que piensas, te conviertes.}\\
\end{minipage}\\

Buda\\
\end{flushright}

\vspace{1cm}

En este último capítulo se van a detallar todos los objetivos y requisitos cumplidos, así como las competencias adquiridas a lo largo de todo el proyecto. Finalmente se describen algunas posibles líneas futuras que pueden dar continuidad a este proyecto.


\section{Objetivos y requisitos cumplidos}

A continuación se van a explicar todos los objetivos y requisitos cumplidos a lo largo de la realización del presente trabajo fin de grado. 

\subsection{Objetivos}

Se ha conseguido cumplir con el objetivo principal de este trabajo fin de grado; esto es, crear un robot que, usando materiales de bajo coste, sea capaz de navegar por las carreteras, detectar los baches que vaya encontrando y sea capaz de estimar el área del bache para hacer una estimación media del volumen que ocupa dicho bache y poder ser tapado; todo ello manipulable mediante una interfaz web. A su vez, se han cumplido todos los subobjetivos establecidos en la Sección \ref{sec:descripcion}:

\begin{enumerate}
	\item Se han investigado las soluciones robóticas actuales al respecto, encontrándose muchas de ellas en fase todavía de experimentación.
	\item Se han seleccionado los componentes \textit{hardware} más adecuados para el robot, siendo  los de menor coste, los elegidos. Además, se han sugerido distintas opciones para abaratar los costes, así como consejos para facilitar el diseño y robustez del robot. 
	\item Se han analizado las diferentes opciones de diseño, que gracias a los bocetos y maquetas, han permitido ir perfilando la idea del diseño hasta encontrar la solución final.
	\item Se ha usado FreeCAD como herramienta libre de diseño CAD.
	\item Como se ha explicado en las Secciones \ref{sec:impresionmontaje} y \ref{sec:expimpresion3d}, en todas las pruebas de impresión se han usado materiales comunes de impresión 3D: tanto PLA como ABS.
	\item Se ha desarrollado  y construido el robot tanto para simulación, usando URDF y ROS 2 Control, como en la vida real, configurando cada componente a bajo nivel.
	\item Al robot se le ha dotado de nodos en ROS2 para conseguir aplicar el objetivo principal en la vida real.
	\item Para demostrar la viabilidad del proyecto, se han realizado experimentos de cada uno de los componentes, así como se ha proporcionado dos posibles aplicaciones completas para poder usar el robot.  
\end{enumerate}

\subsection{Requisitos}

También hay que destacar que se han satisfecho todos los requisitos propuestos en la Sección \ref{sec:requisitos}:

\begin{enumerate}
	\item El coste total de la fabricación del robot ha sido de 237,12€, siendo 250€ el límite establecido.
	\item Todas las piezas diseñadas están pensadas para ser impresas en cualquier impresora 3D, habiéndose probado los diseños en dos impresoras distintas.
	\item Se ha usado el sistema operativo Ubuntu con soporte a largo plazo, en distintas versiones, tanto para el ordenador como para el robot.
	\item Para el entrenamiento de los modelos, se ha usado Google Colaboratory, y así se cumple el requisito de no ser necesario disponer de ninguna tarjeta gráfica de uso dedicado para conseguirlo.
	\item Los modelos entrenados se han convertido al formato necesario para conseguir que funcione a pesar de las limitaciones hardware de Raspberry Pi y la PiCamera.
	\item Para todo el proyecto se ha usado ROS 2 para que sea un proyecto reutilizable por una amplia comunidad de vecinos.
\end{enumerate}

\section{Habilidades desarrolladas}

Además de todas las competencias nombradas en la Sección \ref{sec:competencias}, se han desarrollado numerosas habilidades y conocimientos, de los cuáles a destacar: 

\begin{itemize}
	\item Se han adquirido conocimientos muy profundos sobre cómo investigar sobre un tema y cómo llevarlo a cabo.
	\item Se han adquirido conocimientos profundos de la herramienta FreeCAD.
	\item Se han adquirido conocimientos sobre los parámetros a modificar en una impresión 3D y las necesidad que esta necesita (ventanas cerradas, uso de laca, etc).
	\item Se han adquirido conocimientos sobre mecánica y ensamblaje de piezas. 
	\item Se ha adquirido la habilidad de construir un robot en simulador.
	\item Se han adquirido conocimientos profundos sobre ROS2 Control.
	\item Se ha adquirido el conocimiento de poder configurar y conocer a bajo nivel todos los componentes \textit{hardware} que forman parte del robot.
	\item Se han adquirido conocimientos profundos sobre direcciones IP, conexiones SSH y derivados.
	\item Se ha adquirido la habilidad de implementar un modelo de aprendizaje supervisado sobre un dispositivo con cierta, limitación computacional, como es la Raspberry Pi. 
	\item Se han adquirido conocimientos muy profundos sobre el modelado de un sensor cámara y cómo extraer información de ella para finalmente conocer el área del bache. 
	\item Se han adquirido habilidades sobre desarrollo de páginas web y cómo se comunican con ROS 2.
	\item Se ha adquirido la habilidad de ser capaz de redactar y generar documentación de calidad en un lenguaje de confección de documentos científicos como es LaTeX.
\end{itemize}
 


\section{Líneas futuras}

Finalmente, para permitir la continuidad de este proyecto, se han planteado una serie de líneas futuras a tener en cuenta: 

\begin{itemize}
	\item Dar soporte software al robot en simulación. 
	\item Integrar, configurar y dar soporte software al motor de la cámara.
	\item Modificar la altura de la cámara para conseguir mayor alcance.
	\item Continuar dando soporte a PiBotJ para mantenerlo actualizado en la última versión.
	\item Conseguir implantar a PiBotJ como ayudante real de mantenimiento de carreteras.
\end{itemize}

