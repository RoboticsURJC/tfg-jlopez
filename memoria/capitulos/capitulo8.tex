\chapter{Conclusiones}
\label{cap:capitulo8}

\begin{flushright}
\begin{minipage}[]{10cm}
\emph{La mente lo es todo. En lo que piensas, te conviertes}\\
\end{minipage}\\

Buda\\
\end{flushright}

\vspace{1cm}

En este último capítulo se van a detallar todos los objetivos y requisitos cumplidos; así como las competencias adquiridas a lo largo de todo el proyecto. Finalmente se concluirá la memoria exponiendo las posibles líneas futuras para el proyecto.


\section{Objetivos y requisitos cumplidos}

A continuación se van a explicar todos los objetivos y requisitos cumplidos a lo largo de la realización del presente trabajo fin de grado. 

\subsection{Objetivos}

Es importante destacar que se ha conseguido cumplir con el objetivo principal de este trabajo fin de grado: crear un robot que, usando materiales de bajo coste, sea capaz de navegar por las carreteras, detectar los baches que vaya encontrando y sea capaz de estimar el área del bache para hacer una estimación media del volumen que ocupa dicho bache y poder ser tapado; todo ello registrado sobre una interfaz web. A decir verdad, se ha cumplido cada uno de los subobjetivos establecidos en la Sección \ref{sec:descripcion}:

\begin{enumerate}
	\item Se han investigado las soluciones robóticas actuales, encontrándose muchas de ellas en fase todavía de experimentación.
	\item Se han seleccionado los componentes \textit{hardware} más adecuados para el robot, siendo  los elegidos, los de menor coste. Además, se ha sugerido distintas opciones para abaratar los costes así como, consejos para facilitar el diseño y robustez del robot. 
	\item Se han analizado las diferentes opciones de diseño que gracias a los bocetos y maquetas, han permitido ir perfilando la idea del diseño hasta encontrar la solución final.
	\item Se ha usado FreeCAD como herramienta de diseño \acs{CAD}, conocida como una de las herramientas de diseño libre. 
	\item Como se ha explicado tanto en las Secciones \ref{sec:impresionmontaje} y \ref{sec:expimpresion3d}, en todas las pruebas de impresión se han usado materiales típicos de impresión 3D: tanto PLA como ABS.
	\item Se ha desarrollado  y contruido el robot tanto para simulación, usando URDF y ROS 2 Control, como en la vida real, configurando cada componente a bajo nivel.
	\item Al robot se le ha dotado de nodos en ROS2 para conseguir aplicar el objetivo principal en la vida real.
	\item Para demostrar la viabilidad del proyecto, se han realizado experimentos de cada uno de los componentes, así como se ha proporcionado dos posibles aplicaciones completas para poder usar el robot.  
\end{enumerate}

\subsection{Requisitos}

Tammbién hay que destacar que se han satisfecho todos los requisitos propuestos en la Sección \ref{sec:requisitos}:

\begin{enumerate}
	\item El coste total de la fabricación del robot no debe superar los 250€.
	\item Todas las piezas diseñadas deben poderse imprimir en cualquiera impresora convencional.
	\item Se usará Ubuntu con soporte a largo plazo como sistema operativo, tanto para el ordenador como para el robot.
	\item A fin de facilitar la implementación de este proyecto para cualquier tipo de usuario, no será necesario disponer de ninguna tarjeta gráfica de uso dedicado para entrenar los modelos. 
	\item Los modelos entrenados se deben ajustar a las limitaciones hardware del robot.
	\item Se busca que sea un proyecto a largo plazo, por eso se debe realizar la integración con la plataforma ROS 2. 
\end{enumerate}


%\begin{enumerate}
%	\item{} El coste total de la fabricación del robot no debe superar los 250€.
%	\item{} Todas las piezas diseñadas deben poderse imprimir en cualquiera impresora convencional.
%	\item{} Se usará Ubuntu con soporte a largo plazo como sistema operativo, tanto para el ordenador como para el robot.
%	\item{} A fin de facilitar la implementación de este proyecto para cualquier tipo de usuario, no será necesario disponer de ninguna tarjeta gráfica de uso dedicado para entrenar los modelos. 
%	\item{} Los modelos entrenados se deben ajustar a las limitaciones hardware del robot.
%	\item{} Se busca que sea un proyecto a largo plazo, por eso se debe realizar la integración con la plataforma ROS 2. 
%\end{enumerate}

\section{Habilidades desarrolladas}

Las competencias adquiridas a lo largo del proyecto

conocimientos profundos/skills de Freecad (ampliado experiencia)

He aprendido sobre estructuras y diseños

He podido trabajar con la Raspberry usando puerot CSI, Serial (configurado y probado)  que previamente no se había hecho

He conseguido implementar un modelo de aprendizaje supervisado sobre un dispositivo que tenía limitciones computacionales

He aprendido en muy profundidad el funcioanmiento del modelo pin hole 

He aprendido sobre cómo investigar y los pasos a seguir 

He aprendido a generar documentación de calidad para un trabajo al igual que conocer sus etapas y saber respetarlas 

He aprendido nuevas técnicas matemáticas 

He aprendido sobre cómo usar una interfaz web 

He aprendido a integrar ROS2 a un dispositivo los limitaciones

He aprendido a construir un robot definido por urdf/xacro

He aprendido a integrar  una arquitectura ros2 control en un robot

\section{Líneas futuras}


Por último, añade otro par de párrafos de líneas futuras; esto es, cómo se puede continuar tu trabajo para abarcar una solución más amplia, o qué otras ramas de la investigación podrían seguirse partiendo de este trabajo, o cómo se podría mejorar para conseguir una aplicación real de este desarrollo (si es que no se ha llegado a conseguir).


\begin{itemize}
	\item \textit{Integración en la vida real del motor de la cámara} Aumenta la complejidad...
	\item \textit{Aumentar el campo de visión de la cámara} Haciendo más largo el soporte
	\item \textit{Desarrollar software para el robot simulado} Para que las caídas no duelan
	\item \textit{Implantar PiBotJ como ayudante de mantenimiento} primero en pueblos y más adelante en carreteras
	\item \textit{Continuar dando soporte a PiBotJ} para mantenerlo actualizado en las últimas versiones
\end{itemize}\

No olvides dedicar un par de párrafos para hacer un balance global de qué has conseguido, y por qué es un avance respecto a lo que tenías inicialmente. Haz mención expresa de alguna limitación o peculiaridad de tu sistema y por qué es así. Y también, qué has aprendido desarrollando este trabajo.\\
