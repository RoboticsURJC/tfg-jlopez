\chapter{Experimentos}
\label{cap:capitulo7}

\begin{flushright}
\begin{minipage}[]{10cm}
\emph{Toda la vida es un experimento. Cuantos más experimentos hagas, mejor}\\
\end{minipage}\\

Ralph Waldo Emerson\\
\end{flushright}

\vspace{1cm}

Escribe aquí un párrafo explicando brevemente 

\section{}

explicar experimentos impresora del instituto \\

explicar filtro de imagen v1 con los valores de canny puestos a ojo\\

explicar tipos de tflite usadas \\


https://github.com/RoboticsURJC/tfg-jlopez/wiki/Experimentos (diseño carretera y la primera prueba hecha en la calle con las ruedas de ActivityBot)


\section{Corrector ortográfico}

Una vez tengas todo, no olvides pasar el corrector ortográfico de \LaTeX a todos tus ficheros \textit{.tex}. En \texttt{Windows}, el propio editor \texttt{TeXworks} incluye el corrector. En \texttt{Linux}, usa \texttt{aspell} ejecutando el siguiente comando en tu terminal:

\begin{verbatim}
aspell --lang=es --mode=tex check capitulo1.tex
\end{verbatim}
