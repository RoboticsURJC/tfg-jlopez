\chapter{Experimentos}
\label{cap:capitulo7}

\begin{flushright}
\begin{minipage}[]{10cm}
\emph{Toda la vida es un experimento. Cuantos más experimentos hagas, mejor}\\
\end{minipage}\\

Ralph Waldo Emerson\\
\end{flushright}

\vspace{1cm}

\setcounter{footnote}{132} % Establecer la numeración de la siguiente nota al pie

En este capítulo se tratarán los diferentes experimentos llevados a cabo para la realización de este proyecto y que han permitido conseguir y demostrar la consecución del objetivo y subobjetivos definidos en la Seccción \ref{sec:descripcion}.

\section{Impresión 3D}
\label{sec:expimpresion3d}
Aparte de la solución proporcionada en la Sección \ref{sec:impresionmontaje}, se probaron previamente distintas opciones de diseño y de impresión 3D hasta encontrar la mejor opción.

Primero se optó por diseñar la estructura como una pieza única para facilitar el trabajo de ensamblaje para los usuarios, sin embargo el resultado final no fue el esperado (Figura \ref{fig:imfallida}). 

\begin{figure}[ht!]
	\centering
	\begin{minipage}{0.45\linewidth}
		\centering
		\includegraphics[width=\linewidth]{figs/cap7/impresionfallida1.png}
	\end{minipage}
	\hspace{1cm}
	\begin{minipage}{0.40\linewidth}
		\centering
		\includegraphics[width=\linewidth]{figs/cap7/piezav1error.jpeg}
	\end{minipage}
	\caption{Versiones previas de impresión}
	\label{fig:imfallida}
\end{figure}


Después de eso, se tuvo que cambiar el enfoque y se decidió dividirlo en diferentes piezas para intentar solucionar el problema, como se comentó en la Sección \ref{sec:diseñocad}. En ese momento se estaba usando otra impresora distinta a la definida en la Sección \ref{sec:impresionmontaje}, llamada Ultimaker Cura 2+\footnote{\url{https://ultimaker.com/3d-printers/s-series/ultimaker-2-connect/}}y se empleó ABS en vez de PLA; por lo tanto, los parámetros de impresión fueron distintos (Cuadro \ref{cuadro:cimpresion2}) pero se producía \textit{warping}. Para conocer más acerca de distintas pruebas, puedes consultarlas en la WIKI\footnote{\url{https://github.com/RoboticsURJC/tfg-jlopez/wiki/HARDWARE\#impresi\%C3\%B3n-3d}}.

\begin{table}[H]
	\begin{center}
		\begin{tabular}{|c|c|}
			\hline
			Características & Parámetros\\
			\hline
			 Densidad & 1.04 g/cm³\\
			\hline
			Diámetro & 2.85 mm\\
			\hline
			Temperatura de impresión & 240 ºC\\
			\hline
			Temperatura de la placa & 110 ºC\\
			\hline
			Temperatura en modo de espera & 175 ºC\\
			\hline
			Distancia de retracción & 6.50 mm\\
			\hline
			Velocidad de retracción & 40mm/s\\
			\hline
			Velocidad del ventilador & 10\%\\
			\hline
		\end{tabular}
		\caption{Características usadas en impresiones previas}
		\label{cuadro:cimpresion2}
	\end{center}
\end{table}



\section{Ruedas}
\label{sec:expruedas}
Como se comentó en la Sección \ref{subsec:ruedas}, en este proyecto se han usado 2 tipos de ruedas que gracias a las distintas pruebas realizadas, se ha podido definir las características de cada una de ellas. 

En el primer video\footnote{\url{}} (Figura \ref{fig:pruebaruedas} izquierda) se puede ver a PiBotJ usando las ruedas negras y teniendo dificultades en superficies con irregularidades. Por otro lado, en el segundo video\footnote{\url{}} (Figura \ref{fig:pruebaruedas} derecha) se puede ver a PiBotJ usando las ruedas azules genéricas y superando con mayor facilidad las dificultades que con las ruedas del kit \textit{ActivityBot}.

%% Incluir 2 vídeos y las dos fotos (capturas del video)

%\begin{figure}[ht!]
%	\centering
%	\begin{minipage}{0.4\linewidth}
%		\centering
%		\includegraphics[width=\linewidth]{figs/cap7/pruebanegra.png}
%	\end{minipage}
%	\hspace{1cm}
%	\begin{minipage}{0.40\linewidth}
%		\centering
%		\includegraphics[width=\linewidth]{figs/cap7/pruebaazul.png}
%	\end{minipage}
%	\caption{Versiones previas de impresión}
%	\label{fig:pruebaruedas}
%\end{figure}


\section{Aprendizaje automático}
\label{sec:expaa}
Para la detección de baches, como se comentó... (pequeña intro )

Para conseguir el modelo entrenado en formato .pb, se realizaron numerosas pruebas pero finalmente para la etapa de entrenamiento se decidió usar la siguiente configuración (Figura X); está completamente recogido y facilitado en el fichero args.yaml\footnote{\url{}}.

\begin{itemize}
	\item \verb|detect.tflite|\footnote{\url{https://github.com/RoboticsURJC/tfg-jlopez/blob/main/code/ros2/src/pibotj_rr/custom_model_lite/detect.tflite}}. Esta primera versión una se realizó usando un dataset de 75 imágenes que usa CPU y va lenta, unos 4 segundos de retraso (latencia...)
	\item \verb|detect_edgetpu.tflite|\footnote{\url{https://github.com/RoboticsURJC/tfg-jlopez/blob/main/code/ros2/src/pibotj_rr/custom_model_lite/detect_edgetpu.tflite}} es la primera versión que se realizó intentando convertirla para que fuera compatible con el Google Coral pero no funcionó.
	
	\item \verb|best_full_integer_quant_edgetpu.tflite|\footnote{\url{https://github.com/RoboticsURJC/tfg-jlopez/blob/main/code/ros2/src/pibotj_rr/custom_model_lite/best_full_integer_quant_edgetpu.tflite}} es la segunda versión entrenada convertida para que google coral la detectara. Esta vez sí la detecta pero el modelo era demasiado grande ya que usaba imágenes de 640x480. 
	
	\item \verb|bestv2_full_integer_quant_edgetpu.tflite|\footnote{\url{https://github.com/RoboticsURJC/tfg-jlopez/blob/main/code/ros2/src/pibotj_rr/custom_model_lite/bestv2_full_integer_quant_edgetpu.tflite}} esta es la versión actual usada en el proyecto, convertida a Googel Coral pero usa imágenes de 192x192. La latencia es como mucho de 1 segundo. De puede decir que el problema se ha resuelto con creces. ES muy difícil conseguir real time con las circunstancias presentes. (LA ACTUAL USADAA EN EL PROYECTO) 	
\end{itemize}\


\subsection{Aplicación del modelo}
\label{subsec:expcapmodelo}
Para evitar los picos se usa la media movil exponencial, como se explica en el apartado X y se puede ver cómo esos picos desapaceren gracias a la Gráfica X.

Estudiar gráfica

\section{Modelo pinhole}
\label{sec:expmodelopinhole}
Video con lo del punto rosa y la cartulina blanca


\section{Método shoelace}
\label{sec:expshoelace}
Bache en su dimensión máxima y ver el valor obtenido 

rectángulo y luego demostrar los valores obtenidos

\section{VFF}
\label{sec:expvff}
Mostrar un video cómo se produce la repulsión

\section{Detectar líneas}
\label{sec:expdetectarlineas}
Video de cómo detecta las líneas (con la velocidad original NO CON LOS VALORES DESPACIOS DEL VIDEO ANTIGUO) y el filtro antiguo, con valores puestos a ojo

\section{Interfaz web}
\label{sec:interfazweb}
Reactividad al mostrar bache y cómo se actualiza en la interfaz web 

\section{Ejecución completa de la aplicación}
\label{sec:expcompleto}
Contar cada launcher creado 
\subsection{Teleoperado}
Video  y capturas del ejemplo creado \verb|ros2 launch pibotj_rr robot_teleop.launch.py|
\subsection{Autónomo}
Vídeo y capturas del ejemplo creado \verb|ros2 launch pibotj_rr robot_vff.launch.py|


Cómo se enlazan internamente 
(diagrama de todos los nodos creados) y meter capturas de pantalla !! IMPORTANTE



\section{Snippets}

Puede resultar interesante, para clarificar la descripción, mostrar fragmentos de código (o \textit{snippets}) ilustrativos. En el Código \ref{cod:codejemplo} vemos un ejemplo escrito en \texttt{C++}.

\begin{code}[h]
	\begin{lstlisting}[language=C++]
		void Memory::hypothesizeParallelograms () {
			for(it1 = this->controller->segmentMemory.begin(); it1++) {
				squareFound = false; it2 = it1; it2++;
				while ((it2 != this->controller->segmentMemory.end()) && (!squareFound)) {
					if (geometry::haveACommonVertex((*it1),(*it2),&square)) {
						dist1 = geometry::distanceBetweenPoints3D ((*it1).start, (*it1).end);
						dist2 = geometry::distanceBetweenPoints3D ((*it2).start, (*it2).end);
					}
					// [...]
				\end{lstlisting}
				\caption[Función para buscar elementos 3D en la imagen]{Función para buscar elementos 3D en la imagen}
				\label{cod:codejemplo}
			\end{code}
			
			En el Código \ref{cod:codejemplo2} vemos un ejemplo escrito en \texttt{Python}.
			
			\begin{code}[h]
				\begin{lstlisting}[language=Python]
					def mostrarValores():
					print (w1.get(), w2.get())
					
					master = Tk()
					w1 = Scale(master, from_=0, to=42)
					w1.pack()
					w2 = Scale(master, from_=0, to=200, orient=HORIZONTAL)
					w2.pack()
					Button(master, text='Show', command=mostrarValores).pack()
					
					mainloop()
				\end{lstlisting}
				\caption[Cómo usar un Slider]{Cómo usar un Slider}
				\label{cod:codejemplo2}
			\end{code}
			
			\section{Verbatim}
			
			Para mencionar identificadores usados en el código ---como nombres de funciones o variables--- en el texto, usa el entorno literal o verbatim \verb|hypothesizeParallelograms()|. También se puede usar este entorno para varias líneas, como se ve a continuación:
			
			\begin{verbatim}
				void Memory::hypothesizeParallelograms () {
					// add your code here
				}
			\end{verbatim}
			
			\section{Ecuaciones}
			
			Si necesitas insertar alguna ecuación, puedes hacerlo. Al igual que las figuras, no te olvides de referenciarlas. A continuación se exponen algunas ecuaciones de ejemplo: Ecuación \ref{ec:ec1} y Ecuación \ref{ec:ec2}.
			
			\begin{myequation}[h]
				\begin{equation}
					H = 1 - \frac{\sum_{i=0}^{N}\frac{(\frac{d_{j_s} + d_{j_e}}{2})}{N}}{M}
					\nonumber
					\label{ec:ec1}
				\end{equation}
				\caption[Ejemplo de ecuación con fracciones]{Ejemplo de ecuación con fracciones}
			\end{myequation} 
			
			\begin{myequation}[h]
				\begin{equation}
					v(entrada)= \left\{
					\begin{array}{lcc}
						0 & \mbox{if} & \epsilon_t < 0.1\\
						K_p\cdot{(T_{t}-T)} & \mbox{if}& 0.1 \leq \epsilon_t < M_t\\
						K_p \cdot M_t & \mbox{if}& M_t < \epsilon_t
					\end{array}
					\right.
					\label{ec:ec2}
				\end{equation}
				\caption[Ejemplo de ecuación con array y letras y símbolos especiales]{Ejemplo de ecuación con array y letras y símbolos especiales}
			\end{myequation}
			
			\section{Tablas o cuadros}
			
			Si necesitas insertar una tabla, hazlo dígnamente usando las propias tablas de \LaTeX, no usando pantallazos e insertándolas como figuras... En el Cuadro \ref{cuadro:ejemplo} vemos un ejemplo.
			
			\begin{table}[H]
				\begin{center}
					\begin{tabular}{|c|c|}
						\hline
						\textbf{Parámetros} & \textbf{Valores} \\
						\hline
						Tipo de sensor & Sony IMX219PQ[7] CMOS 8-Mpx \\
						Tamaño del sensor & 3.674 x 2.760 mm (1/4" format) \\
						Número de pixels & 3280 x 2464 (active pixels) \\
						Tamaño de pixel & 1.12 x 1.12 um \\
						Lente & f=3.04 mm, f/2.0 \\
						Ángulo de visión & 62.2 x 48.8 degrees \\
						Lente SLR equivalente & 29 mm \\
						\hline
					\end{tabular}
					\caption{Parámetros intrínsecos de la cámara}
					\label{cuadro:ejemplo}
				\end{center}
			\end{table}
			
			En los textos puedes poner palabras en \textit{cursiva}, para aquellas expresiones en sentido \textit{figurado}, palabras como \textit{robota}, que está fuera del diccionario castellano, o bien para resaltar palabras de una colección: \textit{(a)} es la primera letra del abecedario, \textit{(b)} es la segunda, etc.\\
			
			Al poner las dos líneas del anterior párrafo, este aparecerá separado del anterior. Si no las pongo, los párrafos aparecerán pegados. Sigue el criterio que consideres más oportuno.
			
			\section{Segunda sección}
			\label{sec:segundaseccion}
			
			No olvides incluir imágenes y referenciarlas, como la Figura \ref{fig:roomba}.
			
			\begin{figure} [h!]
				\begin{center}
					\includegraphics[width=8cm]{figs/roomba}
				\end{center}
				\caption{Robot aspirador Roomba de iRobot.}
				\label{fig:roomba}
			\end{figure}\
			
			Ni tampoco olvides de poner las URLs como notas al pie. Por ejemplo, si hablo de la Robocup\footnote{\url{http://www.robocup.org}}.
			
			\subsection{Números}
			\label{sec:subseccion}
			
			En lugar de tener secciones interminables, como la Sección \ref{sec:robotica}, divídelas en subsecciones.
			
			Para hablar de números, mételos en el entorno \textit{math} de \LaTeX, por ejemplo, $1.5Kg$. También puedes usar el símbolo del Euro como aquí: 1.500\euro.
			
			\subsection{Listas}
			
			Cuando describas una colección, usa \texttt{itemize} para ítems o \texttt{enumerate} para enumerados. Por ejemplo:
			
			\begin{itemize}
				\item \textit{Entorno de simulación.} Hemos usado dos entornos de simulación: uno en 3D y otro en 2D.
				\item \textit{Entornos reales.} Dentro del campus, hemos realizado experimentos en Biblioteca y en el edificio de Gestión.
			\end{itemize}\
			
			\begin{enumerate}
				\item Primer elemento de la colección.
				\item Segundo elemento de la colección.
			\end{enumerate}\
			

