\chapter{Anexo}
\label{cap:capitulo9}

\begin{flushright}
\begin{minipage}[]{10cm}
\emph{Las ideas no duran mucho. Hay que hacer algo con ellas}\\
\end{minipage}\\

Santiago Ramón y Cajal\\
\end{flushright}

\vspace{1cm}

En este anexo se van a tratar los pasos necesarios para conseguir replicar este proyecto.

\section{Construcción del robot} 

%incluir enlace del tutorial de construir el robot en físico


\section{Ejecución del robot en simulación}

Es necesario instalar los siguientes programas: 

\begin{verbatim}
	sudo apt update && sudo apt upgrade
	sudo apt install ros-humble-ros2-control ros-humble-ros2-controllers
	sudo apt install ros-humble-rviz2
	sudo apt install ros-humble-gazebo-ros-pkgs
	sudo apt install ros-humble-xacro ros-humble-robot-state-publisher 
	sudo apt install ros-humble-joint-state-publisher
\end{verbatim}
 
Una vez instalado los programas, es el momento de ponerlo en ejecución y para ello, únicamente hay que escribir los siguientes comandos:
\begin{verbatim}
	colcon build --packages-select pibotj_r2c   # compila los paquetes
	source ./install/setup.bash                 # configura variables 
	ros2 launch pibotj_r2c launch_sim.launch.py
\end{verbatim} 

Si la primera vez que se lance el robot ocurre algún error, es normal y hay que reiniciar el proceso.

\section{Configuración del robot real}

