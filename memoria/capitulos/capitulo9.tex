\chapter{Anexo}
\label{cap:capitulo9}

\begin{flushright}
\begin{minipage}[]{10cm}
\emph{Las ideas no duran mucho. Hay que hacer algo con ellas}\\
\end{minipage}\\

Santiago Ramón y Cajal\\
\end{flushright}

\vspace{1cm}

En este apartado se va a tratar los pasos necesarios para conseguir replicar este proyecto 

%incluir enlace del tutorial de construir el robot en físico

\section{Simulación}

\begin{verbatim}
	sudo apt update && sudo apt upgrade
	sudo apt install ros-humble-ros2-control ros-humble-ros2-controllers
	sudo apt install ros-humble-rviz2
	sudo apt install ros-humble-gazebo-ros-pkgs
	sudo apt install ros-humble-xacro ros-humble-robot-state-publisher 
	sudo apt install ros-humble-joint-state-publisher
\end{verbatim}

Una vez instalado todos los programas, fue el momento de empezar a desarrollar el código. 

\subsection{Ejecución en simulador}

Para poner en funcionamiento el modelo, únicamente había que escribir los siguientes comandos:
\begin{verbatim}
	colcon build --packages-select pibotj_r2c   # compila los paquetes
	source ./install/setup.bash                 # configura variables 
	ros2 launch pibotj_r2c launch_sim.launch.py
\end{verbatim} 

Si la primera vez que se lance el robot ocurre algún error, es normal y hay que reiniciar el proceso.


\section{Conclusiones}

Enumera los objetivos y cómo los has cumplido.\\

Enumera también los requisitos implícitos en la consecución de esos objetivos, y cómo se han satisfecho.\\

No olvides dedicar un par de párrafos para hacer un balance global de qué has conseguido, y por qué es un avance respecto a lo que tenías inicialmente. Haz mención expresa de alguna limitación o peculiaridad de tu sistema y por qué es así. Y también, qué has aprendido desarrollando este trabajo.\\

Por último, añade otro par de párrafos de líneas futuras; esto es, cómo se puede continuar tu trabajo para abarcar una solución más amplia, o qué otras ramas de la investigación podrían seguirse partiendo de este trabajo, o cómo se podría mejorar para conseguir una aplicación real de este desarrollo (si es que no se ha llegado a conseguir).

\section{Corrector ortográfico}

Una vez tengas todo, no olvides pasar el corrector ortográfico de \LaTeX a todos tus ficheros \textit{.tex}. En \texttt{Windows}, el propio editor \texttt{TeXworks} incluye el corrector. En \texttt{Linux}, usa \texttt{aspell} ejecutando el siguiente comando en tu terminal:

\begin{verbatim}
aspell --lang=es --mode=tex check capitulo1.tex
\end{verbatim}
